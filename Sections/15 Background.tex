\section{Theoretical background}
\label{sec:background}
In order to understand the rest of the paper, there are some concepts that are important to know about. First, digital technology and AI are introduced at a conceptual level. The notion of stakeholders is discussed, and how they can be included in the design and development of projects.Due to the connections between responsible AI and responsible innovation, a brief introduction to Responsible Research and Innovation is given, before the section rounds out by giving a basic introduction to ethics, mostly focused on applied ethic.

\subsection{Digital technology}
Digital technology is the use of software, hardware and networking technologies, i.e. computers, to digitally process information \parencite{Weisha_2021_digtech}. Such digitization has the possibility of connecting people and organisations across wide geographical distances \parencite{Forman_2019_digtech}, and has impacted a wide range of services, ranging from diabetes awareness campaigns \parencite{OMara_2012_digtech} to energy sustainability \parencite{Wang_2022_digtech} and art education \parencite{Wang_2020_digtech}. 

As digital technology has been shown to increase organisational innovation \parencite{Weisha_2021_digtech}, efficiency \parencite{Grober_2020_digtech} and revenue \parencite{Mohapatra_2022_digtech}, a wide range of organisations have adopted and applied such technologies to their business. To successfully integrate digital technology into an organisation often require organisational changes, a process that is commonly
referred to as digital transformation \parencite{Kretschmer_2020_digtech}. 

Digital transformation is built on five building blocks, as described by \textcite{ross_2018_digtech}. First, organisations looking to adapt digital technology need to have an operational backbone, a set of standardized systems and processes that lets digital technology cooperate, thus increasing efficiency of the organisation. Secondly, organisations need a digital platform, a set of reusable components, facilitating rapid development and experimentation. To support their network, organisations should create an external developer platform, providing easy access for external partners to access and contribute to the organisations' digital technology. In order to get value from these digital platforms, organisations should ensure they have shared customer insights, i.e., a knowledge of what customers are willing to pay for, and an accountability framework, to minimize hierarchy and facilitate rapid delivery of digital technology.

Recent developments of digital technology have led to rapid improvements in the efficiency and flexibility of digital systems, leading some researchers to refer to the current state of digital technology as Industry 4.0 -- the Fourth Industrial Revolution \parencite{Lasi_2014_AIbackground}. This term covers a wide range of modern digital technology, mostly revolving around automation and digitization \parencite{Lu_2017_AIbackground}, including technology such as cloud computing, i.e., off-site servers and computers available for rent; the Internet of Things, i.e., the use of distributed and interconnected sensors, tools and machines that can collect data and communicate with each other to automate and optimize processes \parencite{Xu_2018_AIbackground}; cyber-physical systems, i.e., the interconnectedness of physical machines and digital systems, giving digital technology a way to interact with a physical environment \parencite{Lasi_2014_AIbackground}; mobile computing, i.e., smart phones and portable devices; big data, i.e., the use of large-scale collection and storage of data; and artificial intelligence \parencite{Lu_2017_AIbackground}.


\subsubsection{Artificial intelligence}
% AI / digital technology (more about the phenomenon, why is it important, etc.)
Artificial intelligence has been referred to as \textquote{the next evolution of digital transformation} \parencite{Graham_2020_AIbackground}. Where standard digital technology has been used for diabetes awareness, energy sustainability and art education, AI has been used to predict the risk of developing diabetes \parencite{Ellahham_2020_AIbackground}, has been named an enabler of 134 of the United Nations Sustainable Development Goals \parencite{Vinuesa_2020_AIbackground}, and has been used to make systems capable of autonomously creating art \parencite{Cetinic_2022_AIbackground}.

As AI is a form of digital transformation, the same building blocks, described by \textcite{ross_2018_digtech}, must be in place to succeed with AI projects, with some adaptations. As AI learns from data \parencite{Graham_2020_AIbackground}, organisations looking to successfully adopt AI technologies must ensure their operational backbone and digital platforms are designed to standardize and systematize data collection and storage. This way, AI technologies can be developed in the same rapid fashion as other digital technology, without having to repeat the process of data collection each time \parencite{Werder_2022}. Additionally, in scenarios where such data may contain personal information, organisations should expand the notion of shared customer insights. Where ordinary digital transformation only needs insight into what digital services a customer is willing to pay for, organisations looking to adopt AI technologies should also consider how the process of collecting and using data, as well as the data itself, may affect customers and noncustomers alike \parencite{Dignum_2021}.

Artificial intelligence is further explored and defined in \autoref{sec:definition-ai}.


\subsection{Stakeholders}
The notion of stakeholders, whose fame is often attributed to  \citeauthor{Freeman_1984_stakeholders,Freeman_1983_stakeholders} (\citeyear{Freeman_1983_stakeholders,Freeman_1984_stakeholders}; see, e.g., \cite{Mitchell_1997_stakeholders}), is that companies have obligations for other groups than only those holding shares in the company \parencite{Freeman_1983_stakeholders}. When making decisions, organisations should therefore not only consider what impact a decision may have on the economy of the organisation, but also what effect it may have on other affected people \parencite{Freeman_1983_stakeholders}. These term stakeholders, then, is used to represent these other affected people.

\textcite{Mitchell_1997_stakeholders} note that there have been multiple, somewhat contradictory definitions of stakeholders throughout history, and that these mostly differ along two axes. The first axis is that of broadness, where definitions range from narrow ones, such as \textquote{Any identifiable group or individual on which the organisation is dependent for its continued survival} \parencite{Freeman_1983_stakeholders}, to broad definitions like \textquote{any group or individual who can affect or is affected by the achievement of the organization's objectives} (\cite{Freeman_1984_stakeholders}; as cited in \cite[p.~856]{Mitchell_1997_stakeholders}). The second axis is that of primary versus secondary stakeholders, where primary stakeholders are typically defined as those who bear risk through commitment of resources, e.g., shareholders, employees, customers, suppliers and the environment \parencite{Hillman_2001_stakeholders}, while secondary stakeholders are those who can influenced and are influenced by an organisation, but that the company can survive without \parencite{Benn_2016_stakeholders}. As it is outside of the scope of this paper to conduct a deep-dive into definitions of stakeholders, interested readers are guided to see the work of \textcite{Mitchell_1997_stakeholders}.

Who should be considered as stakeholders vary based on the context and nature of a project, but there exists several universal tools for identifying these. \textcite{Reed_2009_stakeholders} propose three methods for identifying stakeholders. First, organisations may use domain experts to provide stakeholders based on their experiences with similar projects. This method requires few resources, but has the greatest potential for bias among the methods. Secondly, organisations may conduct focus group sessions, where a small group of internal and external parties brainstorm potential stakeholders for a project. Finally, once some stakeholders have been identified, organisations may use snowball sampling, i.e., interviewing already identified stakeholders, to identify new categories. These methods can be used alongside each other to the number of relevant stakeholders that are identified. Once all potential stakeholders have been identified, organisations may use stakeholder maps to categorize them, giving a visual overview of all stakeholders, their importance for the project and what communication strategies are most likely to give good results for each stakeholder group \parencite{Hoory_2022_stakeholders}.


\subsection{Responsible Research and Innovation}
Responsible Research and Innovation (RRI) is a process of innovating based not only on what provides the largest economic value, but combining economic gain with ethical values and societal needs to develop products and services that benefit society as a whole \parencite{Werker_2020_RRI}. The term was first used in 2013, but has gained popularity after its inclusion in the European Commission's Horizon 2020 Framework Program \parencite{Burget_2017_RRI}.

Two core methods exist to ensure research and innovation processes are conducted in a responsible way. First, \textcite{Werker_2020_RRI} highlights the need to include stakeholders, as they are needed to fully understand the effects of research processes -- and potential alternatives that can be pursued. The definition of stakeholders should be broad, to ensure as many people as possible are included, and the collaboration should be an ongoing process, where the stakeholders actively take part in the research project throughout its life cycle \parencite{Schomberg_2013_RRI}.

Secondly, responsible research projects should be developed according to a set of conceptual dimensions \parencite{Burget_2017_RRI}. Although many different dimensions exist, \textcite{Burget_2017_RRI} finds that the dimensions Anticipation, i.e., the ability to envision the future of research, and how work can change the future; Inclusion, i.e., the early inclusion of stakeholders; Responsiveness, i.e., the ability to respond to risk and act in a transparent way; Reflexivity, i.e., the ability to reflect on limitations of research, and how it may affect people with different backgrounds; Sustainability, i.e., ensuring resource-efficiency is increased with new innovation, and doing so in a sustainable way; and Care, i.e., the ability to emphasize, and get stakeholders to care about responsible research, are frequently discussed in RRI literature, and present a good framework for ensuring research is conducted in a responsible way.


\subsection{Ethics}
Ethics revolve around questions of what is right and wrong, and what actions is "correct" for a given setting. The field is typically divided into three branches -- meta-ethics, which looks at the language and sources of ethics; normative ethics, which looks at norms and standards for action; and applied ethics, which looks at how these norms and standards can be implemented in given contexts \parencite{Clarke_2019}. As the field of responsible AI looks at both norms and standards for AI development and how it can be applied to real development scenarios, the research being done on responsible AI falls under both normative and applied ethics.

As the ethics is a massive field of research, delving too deep into ethical theory is outside the scope of this paper. Still, this surface-level introduction should be enough to understand the ethics terms used throughout the paper.

% According to \textcite{Allhoff_2011_ethics}, applied ethics approaches can generally be separated into three categories -- top-down, bottom-up and reflexive approaches.