\section{Conclusion}
\label{sec:Conclusion}
\todo[inline]{Very temporary}


%%% Future work %%%
\subsection{Further work}
% A complete taxonomy of the field of AI ethics (all alternative terms for responsible AI)
% Obviously rewrite
This paper is focused on responsible AI. While alternative terms for responsible AI are mentioned and briefly discussed in \autoref{sec:results-alernative-terms}, delving too deep into the many, somewhat overlapping, terms for topics related to responsible AI is outside of the scope for this paper. To unify this wide range of terms, there is a need for an ontology or 
\todo{Should I focus on one of those? I am not fully clear on the difference}
taxonomy of the field. Ontologies are \textquote{representation vocabularies}, capturing the conceptualizations underlying terms \parencite{Chandrasekaran_1999} whereas taxonomies \textquote{help humans classify objects according to similarities and differences} \parencite{Kundisch_2022}. Creating an ontology or taxonomy is thus a method to formally create a unifying definition of terms \parencite{Uschold_1996}, and would help to harmonize the wide usage of different terms within the field of AI ethics. Although AI has been used to aid ontology creation \parencite{Stumme_2001} and significant efforts have gone into creating taxonomies of explainable AI \parencite{BarredoArrieta_2020,Clinciu_2019,Graziani_2022,Bellucci_2021,Brennen_2020}, as well as specific uses of AI \parencite{Gunn_2009,Ören_1994,Fong_2003}, there is a lack of a unifying ontology or taxonomy, and thus a lack of common, agreed-upon definitions, within responsible AI or the wider field of AI ethics. % Something to conclude how it would help?

\autoref{sec:definition-ai} does some work towards creating a unified definition of artificial intelligence. While a deep-dive into definitions and terminology is outside of the scope of this paper, there appears to be a lack of consensus regarding how to define the concept of artificial intelligence, which should be considered the backbone of responsible artificial intelligence. As such, AI researchers should work together with researchers within the field of philosophy and language to create a unifying, agreed-upon definition of artificial intelligence. This, much like the above-mentioned taxonomy, would align future research along agreed-upon lines, thus ensuring work is moving the field in the same direction, while limiting disagreements and confusion.

% Connections between the field of ethics (i.e., non-AI ethics, ethical theory and philosophy) and responsible AI?

% Consider ways to democratically create frameworks / guidelines. Create guidelines anchored in the general public in a democratic way (Gianni R., Lehtinen S., Nieminen M.)