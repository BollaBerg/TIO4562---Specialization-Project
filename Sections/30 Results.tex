\section{Findings}
\label{sec:Results}
% \subsection{Descriptive Results} % See Lukkien et al.
\todo[inline]{Descriptive results - number of papers per stage, results from methodology. Include any relevant observations -- distribution of methodology (reviews vs primary (qualitative / quantitative) studies). Consider including an overview of each paper with its context and methodology (and possibly more?)}

% Number of papers per stage, results from the methodology
\begin{figure}[p]
    \centering
    \import{./Images/}{PRISMA-flowchart}
    \caption[Overview of literature review]{Overview of literature review. Adapted from \textcite{PRISMA_2022}.}
    \label{fig:PRISMA-flowchart}
\end{figure}

% While the field is far from mature, 289 (?) unique results in the preliminary search means it is evolved enough for a systematic literature review

% Compare primary vs secondary studies

% Mention non-peer-reviewed papers included in the review, and why they were included

% Cite studies that might appear to meet the inclusion criteria, but which were excluded, and explain why they were excluded.


% \subsection{Goals for responsible AI}
% Have I discovered any goals for responsible AI? What do the reviewed papers say?
%   See table 1 of Barredo Arrieta et al.

\subsection{Key concepts}
\import{Sections/}{35 Definitions}


\subsection{RQ1 -- Principles for responsible AI}
\label{sec:results-rq1}
% Start by defining responsible AI

% Perhaps a (large) table, summarizing the principles from each paper? I.e.
%               |   Explainability  |   Fairness    |   Security
%   Berg (2021) | Interpretability  |   Justice     |   Safety
%   ...

% Alternatively move this ^^ to appendix, and instead have a table:
%   Principle       |   Explanation     |   Used by
%    Explainability |   The ability ... |   Berg (2021), ...
%    Fairness       |   Lack of bias    |   Berg (2021), ...
%    ...

% The principles of the reviewed papers largely belongs to one of three clusters, depending on whether they are based on \textcite{Fjeld_2020}, \textcite{Floridi_2018} or \textcite{Jobin_2019}
\begin{table}[htp]
    \centering
    \begin{tabular}{lp{0.5\textwidth}p{0.3\textwidth}}
        \textbf{Principle} & \textbf{Explanation} & \textbf{Used by} \\
        \midrule
        Accountability & The ability to explain and justify decisions to stakeholders & \textcite{Dignum_2021,Cheng_2021,Mikalef_2022,Anagnostou_2022} \\
        Explainability & Description of explainability, based on some sources & \textcite{BarredoArrieta_2020} \\
        \bottomrule
    \end{tabular}
    \caption{Overview of each principle}
    \label{tab:my_label}
\end{table}

Alternative:
\begin{table}[htp]
    \centering
    \begin{tabular}{ccccc}
        & Accountability & Transparency & Non-maleficence & ... \\
     \midrule
        \textcite{Anagnostou_2022} & Accountability & Transparency & & Fairness \\
        \textcite{BarredoArrieta_2020} & Accountability & Explainability & & \\
        \textcite{Borda_2022} & &  Explicability & Non-maleficence & Justice \\
        ...
    \end{tabular}
    \caption{Overview of each paper's principles}
    \label{tab:my_label}
\end{table}

\subsubsection{Accountability}
...

\subsubsection{Transparency}
...
  
% Then explanation of each principle, citing what the different sources have said, summarizing and possibly defining the principle in the context of RAI
% Include how the principles build upon each other / connections between principles
%   - I.e., how explainability is needed for the other principles to be applied
% Include also ways to enable the different principles, i.e. column "How to enable principles"

% Finally, discuss principles used by sources but not included here, as well as the reason why it was not included

% Possibly include some generic "How to enable principles" advice that apply to multiple principles
%   - E.g. data provenance (Werder K., Ramesh B., Zhang R.S.)

\subsection{RQ2 -- Antecedents for responsible AI}

\subsection{RQ3 -- Business advantages of responsible AI}
% Academic stuff

% Outside of academia... (see Snowballing sources)


\subsection{Alternative terms for responsible AI}
\label{sec:results-alernative-terms}
% A comparison of the different terms for responsible AI I have seen throughout the review, and how they compare to each other.
% Use as base for future research -- a complete taxonomy of the field, establishing common and standardized terms
% Possibly a comparison of SCOPUS hits for the given terms, to show their popularity in comparison to each other

% \subsection{Challenges with responsible AI}
\subsection{Criticisms of responsible AI}
\label{sec:results-criticism}
% What challenges / barriers have the review shown?
% I.e., lack of implementation guides / most guidelines are very general and vague (maybe a "how to implement RAI" subsection?)
%   - Are also found in general ethics: https://www.scopus.com/record/display.uri?eid=2-s2.0-0025411067&origin=resultslist&sort=cp-f&src=s&st1=autonomy+AND+beneficence+AND+justice&nlo=&nlr=&nls=&sid=4734d518162d88ad8f175bc2a82e4e4f&sot=b&sdt=b&sl=41&s=ALL%28autonomy+AND+beneficence+AND+justice%29&relpos=9&citeCnt=324&searchTerm=
% "Ethics washing"
% Lack of quantitative research