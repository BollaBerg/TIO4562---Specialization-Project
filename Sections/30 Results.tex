\section{Results}
\label{sec:Results}
% Number of papers per stage, results from the methodology
\begin{figure}[p]
    \centering
    \import{./Images/}{PRISMA-flowchart}
    \caption[Overview of literature review]{Overview of literature review. Adapted from \textcite{PRISMA_2022}.}
    \label{fig:PRISMA-flowchart}
\end{figure}

% Cite studies that might appear to meet the inclusion criteria, but which were excluded, and explain why they were excluded.

% Definitions (both from review and thought-up myself)
% AI
% Responsible AI

% \subsection{Goals for responsible AI}
% Have I discovered any goals for responsible AI? What do the reviewed papers say?
%   See table 1 of Barredo Arrieta et al.


\subsection{RQ1 -- Principles for responsible AI}
% Start by defining responsible AI

% Perhaps a (large) table, summarizing the principles from each paper? I.e.
%               |   Explainability  |   Fairness    |   Security
%   Berg (2021) | Interpretability  |   Justice     |   Safety
%   ...

% Alternatively move this ^^ to appendix, and instead have a table:
%   Principle       |   Explanation     |   Used by
%    Explainability |   The ability ... |   Berg (2021), ...
%    Fairness       |   Lack of bias    |   Berg (2021), ...
%    ...

% Then explanation of each principle, citing what the different sources have said, summarizing and possibly defining the principle in the context of RAI
% Include how the principles build upon each other / connections between principles
%   - I.e., how explainability is needed for the other principles to be applied

% Finally, discuss principles used by sources but not included here, as well as the reason why it was not included

\subsection{RQ2 -- Antecedents for responsible AI}

\subsection{RQ3 -- Business advantages of responsible AI}


% \subsection{Alternative terms for responsible AI}
% A comparison of the different terms for responsible AI I have seen throughout the review, and how they compare to each other.
% Use as base for future research -- a complete taxonomy of the field, establishing common and standardized terms
% Possibly a comparison of SCOPUS hits for the given terms, to show their popularity in comparison to each other

% \subsection{Challenges with responsible AI}
% \subsection{Criticisms of responsible AI}
% What challenges / barriers have the review shown?
% I.e., lack of implementation guides / most guidelines are very general and vague (maybe a "how to implement RAI" subsection?)
% "Ethics washing"
% Lack of quantitative research