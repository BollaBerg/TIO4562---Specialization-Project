\begin{landscape}
\section{Overview of reviewed papers}
\label{app:overview}


\begin{ThreePartTable}
\begin{TableNotes}
\tiny
\item [*] Not applicable to this paper.
\item [\textdagger] No methodology provided
\end{TableNotes}

\scriptsize

\centering
\begin{longtable}{P{0.2\linewidth}P{0.45\linewidth}P{0.15\linewidth}P{0.15\linewidth}}

    \caption{Overview of the reviewed papers}
    \label{tab:paper-overview} \\
    \toprule
        \textbf{Article} & \textbf{Summary} & \textbf{Context} & \textbf{Methodology} \\
    \midrule
    \endfirsthead
    
    \toprule
        \textbf{Article} & \textbf{Summary} & \textbf{Context} & \textbf{Methodology} \\
    \midrule
    \endhead

    \bottomrule
    \multicolumn{4}{r}{\tiny \textit{Continued on next page\ldots}}
    \endfoot

    \bottomrule
    \insertTableNotes  % tell LaTeX where to insert the contents of "TableNotes"
    \endlastfoot
    
        \textcite{Anagnostou_2022} & Looks at Business management, Transportation, Healthcare, \mbox{E-Government} \& public sector and Information technology, to see what issues these industries face regarding responsible AI, and what principles are needed to handle these issues. & Global, \mbox{multiple sectors} & Literature review \\ 
        \textcite{BarredoArrieta_2020} & Provides an in-depth, technical overview of explainable AI. & N/A\tnote{*} & Literature review \\ 
        \textcite{Bélisle-Pipon_2022} & Reviews existing guidelines for responsible AI, focusing on methodology and \mbox{stakeholder} engagement. & N/A\tnote{*} & Review of guidelines \\ 
        \textcite{Benjamins_2019} & Presents a set of principles for responsible AI, and a methodology for implementing them. & Large, multinational \mbox{organization} & Case study \\ 
        \textcite{Borda_2022} & Looks at how principles for responsible AI can have a positive effect on studies of diseases. & N/A\tnote{*} & Unknown\tnote{\textdagger} \\ 
        \textcite{Brand_2022} & Presents a framework for implementing responsible AI in government. & South Africa, \mbox{government} & Review of guidelines \\ 
        \textcite{Buhmann_2021} & Presents a high-level framework for implementing responsible AI & N/A\tnote{*} & Conceptual \\ 
        \textcite{Canca_2020} & Reviews existing principles for responsible AI, separating them into "core" and "instrumental" principles. & N/A\tnote{*} & Viewpoint \\ 
        \textcite{Chen_2020} & Presents a framework for implementing responsible AI. & N/A\tnote{*} & Conceptual \\ 
        \textcite{Cheng_2021} & Presents a framework for implementing responsible AI. & N/A\tnote{*} & Conceptual \\ 
        \textcite{Clarke_2019} & Reviews existing guidelines for responsible AI, combining them to create 50 \mbox{principles} divided into 10 themes. & N/A\tnote{*} & Review of guidelines \\ 
        \textcite{Dignum_2017} & Presents the ART-principles for responsible AI. & Europe, labor & Expert discussion \\ 
        \textcite{Dignum_2019} & Presents a framework for connecting abstract principles to concrete \mbox{implementations}. & N/A\tnote{*} & N/A\tnote{*} \\ 
        \textcite{Dignum_2021} & Looks at how AI can be used responsibly in education. & Education & Unknown\tnote{\textdagger} \\ 
        \textcite{Doorn_2021} & Reviews how responsible AI can be used in the water domain. & Global, \mbox{water domain} & Literature review \\ 
        \textcite{Eitel-Porter_2021} & Presents a DevOps-look at how responsible AI can be implemented and handled in production systems. & N/A\tnote{*} & Viewpoint \\ 
        \textcite{Fjeld_2020} & Provides detailed descriptions of existing principles for responsible AI. & N/A\tnote{*} & Review of guidelines \\ 
        \textcite{Floridi_2019} & Reviews existing principles for responsible AI, finding a convergence upon a set of common principles. & Global & Review of guidelines \\ 
        \textcite{Floridi_2018} & Presents results from the AI4People initiative, including 20 action points for \mbox{implementing} responsible AI. & N/A\tnote{*} & Expert discussion \\ 
        \textcite{Gianni_2022} & Presents a philosophical view of responsible AI, looking at how it can be achieved in a democratic way. & Global & Review of AI strategies \\ 
        \textcite{Gupta_2021} & Looks at AI in healthcare through a perceived risk theory-lens. & India, healthcare & Quantitative survey \\ 
        \textcite{Hacker_2022} & Looks at current and proposed laws regulating AI in Europe. & Europe, laws and regulation & Review of laws \\ 
        \textcite{Hagendorff_2020} & Draws a connection between how common a principle is, and how \mbox{technical} its solution is. & N/A\tnote{*} & Review of guidelines \\ 
        \textcite{Havrda_2020} & Presents an impact assessment framework to operationalize principles for \mbox{responsible} AI. & N/A\tnote{*} & Conceptual \\ 
        \textcite{Henriksen_2021} & Looks at how developers consider different accountability measures. & Scandinavia, \mbox{AI developers} & Case study \\ 
        \textcite{Jakesch_2022} & Finds that stakeholders and AI developers have differing views on the importance of AI principles. & The US & Quantitative survey \\ 
        \textcite{Jobin_2019} & Reviews existing guidelines for responsible AI, using that to create a new set of principles. & Western countries & Review of guidelines \\ 
        \textcite{Kumar_2021} & Looks at connections between the responsibility of AI and the perceived and actual value of a solution. & India, healthcare & Qualitative interviews, quantitative survey \\ 
        \textcite{Liu_2021} & Reviews existing guidelines for responsible AI, using that to create a new set of principles. & China, healthcare & Qualitative interviews \\ 
        \textcite{Lu_2022} & Presents a detailed list of methods for implementing responsible AI. & N/A\tnote{*} & Conceptual \\ 
        \textcite{Lukkien_2021} & Reviews existing AI principles within the field of long-term healthcare. & Healthcare & Literature review \\ 
        \textcite{Merhi_2022} & Looks at existing barriers preventing AI from being responsible. & N/A\tnote{*} & Literature review, \mbox{qualitative interviews} \\ 
        \textcite{Mikalef_2022} & Presents the authors' view on the direction of future AI research. & N/A\tnote{*} & Viewpoint \\ 
        \textcite{Minkkinen_2021} & Reviews AI strategies in place in Europe, and how that impacts the AI ecosystem. & Europe, \mbox{AI strategies} & Review of AI strategies \\ 
        \textcite{Morley_2020} & Presents a detailed list of tools for integrating ethics in AI systems. & N/A\tnote{*} & Literature review \\ 
        \textcite{Morley_2021} & Looks at barriers preventing AI from being responsible, as well as ways to overcome them. & The UK & Qualitative interviews, quantitative survey \\ 
        \textcite{Nauck_2019} & Presents recommendations for managers interested in making AI \mbox{responsible}. & N/A\tnote{*} & Viewpoint \\ 
        \textcite{Nevanperä_2021} & Reviews existing guidelines. & N/A\tnote{*} & Review of guidelines \\ 
        \textcite{Papagiannidis_2022} & Looks at business advantages that can be gained from making AI \mbox{responsible}. & Nordic countries & Quantitative survey \\ 
        \textcite{Peters_2020} & Presents two frameworks for implementing responsible AI, and applies them to a case study. & Global, healthcare & Case study, \mbox{qualitative interviews} \\ 
        \textcite{Rakova_2021} & Presents a detailed list of organisational changes that enable responsible AI. & Global, \mbox{AI developers} & Qualitative interviews, \mbox{expert workshop} \\ 
        \textcite{Rizinski_2022} & Maps existing ethical challenges of finance to AI ethics. & Global, finance & Qualitative interviews \\ 
        \textcite{Rothenberger_2019} & Presents a set of principles for responsible AI, and uses experts and laypersons to prioritize them. & N/A\tnote{*} & Literature review, \mbox{qualitative interviews}, \mbox{quantitative survey} \\ 
        \textcite{Ryan_2021} & Presents a detailed list of normative recommendations for creating \mbox{responsible} AI. & N/A\tnote{*} & Review of guidelines \\ 
        \textcite{Siala_2022} & Summarizes current work on responsible AI within the field of healthcare. & Global, healthcare & Literature review \\ 
        \textcite{Thelisson_2018} & Presents ethical theories and principles for responsible AI. & N/A\tnote{*} & Unknown\tnote{\textdagger} \\ 
        \textcite{Vakkuri_2022} & Compares actual implementations of responsible AI to theoretical guidelines. & Finland, \mbox{AI developers} & Qualitative interviews \\ 
        \textcite{vanBruxvoort_2021} & Presents a framework for implementing responsible AI. & The Netherlands, government & Case study \\ 
        \textcite{Vetro_2019} & Presents a set of principles for developing responsible AI agents. & N/A\tnote{*} & Unknown\tnote{\textdagger} \\ 
        \textcite{WangW_2021} & Looks at how responsible AI impacts technology adoption. & China, healthcare & Quantitative survey \\ 
        \textcite{WangY_2020} & Draws a connection between responsible AI and corporate social \mbox{responsibility}. & Western countries & Literature review \\ 
        \textcite{Werder_2022} & Looks at ways to responsibly handle data, to use for creating \mbox{responsible} AI. & N/A\tnote{*} & Conceptual \\ 
        \textcite{Wright_2018} & Looks at how automating business processes impact stakeholders. & N/A\tnote{*} & Conceptual \\ 
\end{longtable}

\end{ThreePartTable}

\normalsize

\end{landscape}